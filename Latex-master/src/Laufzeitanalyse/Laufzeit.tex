\section{Laufzeitanalyse}
\subsection{Einleitung}
Um die Effizienz des Programms beurteilen zu können, werde ich die einzelnen Algorithmen betrachten. Dazu werde ich Das erstellen einer Datenbank, berechnen der durchschnittlichen Farbe, skalieren der Bilder und das berechnen der besten Bilder vergleichen. Der Test kann mit einer UI im Programm durchgeführt werden. Es wird ein Test mehrfach durchgeführt und bei jedem neuem Durchgang die Anzahl an Bildern erhöht. Das Testen kann mit generierten Bildern und mit zufälligen Bildern der Internetseite ``https://picsum.photos'' durchgeführt werden. Alle folgenden Tests werden mit zufälligen Bildern durchgeführt, da diese die echte Benutzung am besten Simulieren. Bei den Generierten Bildern sinken die Berechnungszeiten stark, da weniger Farbkomplexität gegeben ist. Bei jeder Simulation wird auch der benötigte Arbeitsspeicher gespeichert, da das Programm sehr Arbeitsspeicher intensiv werden kann. Die Laufzeit des Berechnen der durchschnittlichen Farbe, das erstellen einer Datenbank und das berechnen der besten Bilder ist Linear. $O(n)$ mit der Anzahl der Bilder als $n$. Die Laufzeit des Skalieren ist abhängig von dem genutzten Algorithmus. Da je nach ausgewählter Einstellung an
\begin{figure}[h]
    \centering
    \begin{tikzpicture}
    \begin{groupplot}[
        group style={
        group name=laufzeit,
        group size= 2 by 2,
        horizontal sep =1.5cm,
        vertical sep =2cm},
        width=2in
        ]         
        \nextgroupplot[]
            \addplot table [mark=none,x=x, y=time, col sep=semicolon] {./images/Simulationen/Database_1.000x10.000.csv};
            \addplot table [mark=none,x=x, y=ram, col sep=semicolon] {./images/Simulationen/Database_1.000x10.000.csv};
         
        \nextgroupplot[]
            \addplot table [mark=none,x=x, y=time, col sep=semicolon] {./images/Simulationen/averageColor_500x50_500x500_Ultra.csv};
            \addplot table [mark=none,x=x, y=ram, col sep=semicolon] {./images/Simulationen/averageColor_500x50_500x500_Ultra.csv};
         
        \nextgroupplot[]
            \addplot table [mark=none,x=x, y=time, col sep=semicolon] {./images/Simulationen/scalingImages.csv};
            \addplot table [mark=none,x=x, y=ram, col sep=semicolon] {./images/Simulationen/scalingImages.csv};
         
        \nextgroupplot[]
            \addplot table [mark=none,x=x, y=time, col sep=semicolon] {./images/Simulationen/computation.csv};
            \addplot table [mark=none,x=x, y=ram, col sep=semicolon] {./images/Simulationen/computation.csv};

        \end{groupplot}
         
        % Bildunterschriften
        \tikzset{SubCaption/.style={
        text width=2in,yshift=-3mm, align=center,anchor=north
        }}
         
        \node[SubCaption] at (laufzeit c1r1.south) {\caption Datenbank\label{subplot:eins}};
         
        \node[SubCaption] at (laufzeit c2r1.south) {\caption durchschnittliche Farbe\label{subplot:zwei}};
         
        \node[SubCaption] at (laufzeit c1r2.south) {\caption skalieren der Bilder\label{subplot:drei}};
         
        \node[SubCaption] at (laufzeit c2r2.south) {\caption berechnen der besten Bilder\label{subplot:vier}};
    \end{tikzpicture}
    \caption[BilinearBikubisch]{Vergleich Bilinear und Bikubisch}
\end{figure}
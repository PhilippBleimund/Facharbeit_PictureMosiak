\section{Laufzeitanalyse}
\subsection{Einleitung}
Um die Effizienz des Programms beurteilen zu können, werde ich die einzelnen Algorithmen betrachten. Dazu werde ich Das erstellen einer Datenbank, berechnen der durchschnittlichen Farbe, skalieren der Bilder und das berechnen der besten Bilder vergleichen. Der Test kann mit einer UI im Programm durchgeführt werden. Es wird ein Test mehrfach durchgeführt und bei jedem neuem Durchgang die Anzahl an Bildern erhöht. Das Testen kann mit generierten Bildern und mit zufälligen Bildern der Internetseite ``https://picsum.photos'' durchgeführt werden. Alle folgenden Tests werden mit zufälligen Bildern durchgeführt, da diese die echte Benutzung am besten Simulieren. Bei den Generierten Bildern sinken die Berechnungszeiten stark, da weniger Farbkomplexität gegeben ist. Bei jeder Simulation wird auch der benötigte Arbeitsspeicher gespeichert, da das Programm sehr Arbeitsspeicher intensiv werden kann. Die Laufzeit des Berechnen der durchschnittlichen Farbe, das erstellen einer Datenbank und das berechnen der besten Bilder ist Linear. $O(n)$ mit der Anzahl der Bilder als $n$. Die Laufzeit des Skalieren ist abhängig von dem genutzten Algorithmus.
\begin{figure}[h]
    \centering
    \subfloat[\centering Datenbank]{
        \begin{tikzpicture}[scale = 0.85]
            \begin{axis}[tick label style={
                /pgf/number format/fixed,
                /pgf/number format/fixed zerofill,
                /pgf/number format/precision=1
            }, legend pos=north west, no markers, legend style={nodes={scale=0.7, transform shape}}]
                \addplot table [x=x, y=time, col sep=semicolon] {./images/Simulationen/Database_1.000x10.000.csv};
                \addlegendentry{Zeit in ns}
                \addplot table [x=x, y=ram, col sep=semicolon] {./images/Simulationen/Database_1.000x10.000.csv};
                \addlegendentry{Ram in Byte}
            \end{axis}
        \end{tikzpicture}
        \label{DatenbankGraph}
    }
    \subfloat[\centering durchschnittliche Farbe]{
        \begin{tikzpicture}[scale = 0.85]
            \begin{axis}[tick label style={
                /pgf/number format/fixed,
                /pgf/number format/fixed zerofill,
                /pgf/number format/precision=1
            }, legend pos=north west, no markers, legend style={nodes={scale=0.7, transform shape}}]
                \addplot table [x=x, y=time, col sep=semicolon] {./images/Simulationen/averageColor_500x50_500x500_Ultra.csv};
                \addlegendentry{Zeit in ns}
                \addplot table [x=x, y=ram, col sep=semicolon] {./images/Simulationen/averageColor_500x50_500x500_Ultra.csv};
                \addlegendentry{Ram in Byte}
            \end{axis}
        \end{tikzpicture}
        \label{DurchschnittlicheFarbeGraph}
    }
    \hspace{0mm}
    \subfloat[\centering skalieren der Bilder]{
        \begin{tikzpicture}[scale = 0.85]
            \begin{axis}[tick label style={
                /pgf/number format/fixed,
                /pgf/number format/fixed zerofill,
                /pgf/number format/precision=1
            }, legend pos=north west, no markers, legend style={nodes={scale=0.7, transform shape}}]
                \addplot table [x=x, y=time, col sep=semicolon] {./images/Simulationen/scalingImages.csv};
                \addlegendentry{Bilinear(ns)}
                \addplot table [x=x, y=time, col sep=semicolon] {./images/Simulationen/scalingImagesBikubisch.csv};
                \addlegendentry{Bikubisch(ns)}
                \addplot table [x=x, y=ram, col sep=semicolon] {./images/Simulationen/scalingImages.csv};
                \addlegendentry{Bilinear(Byte)}
                \addplot table [x=x, y=ram, col sep=semicolon] {./images/Simulationen/scalingImagesBikubisch.csv};
                \addlegendentry{Bikubisch(Byte)}
            \end{axis}
        \end{tikzpicture}
        \label{SkalierenGraph}
    }
    \subfloat[\centering berechnen der besten Bilder]{
        \begin{tikzpicture}[scale = 0.85]
            \begin{axis}[tick label style={
                /pgf/number format/fixed,
                /pgf/number format/fixed zerofill,
                /pgf/number format/precision=1
            }, legend pos=north west, no markers, legend style={nodes={scale=0.7, transform shape}}]
                \addplot table [x=x, y=time, col sep=semicolon] {./images/Simulationen/computation.csv};
                \addlegendentry{Zeit in ns}
                \addplot table [x=x, y=ram, col sep=semicolon] {./images/Simulationen/computation.csv};
                \addlegendentry{Ram in Byte}
            \end{axis}
        \end{tikzpicture}
        \label{BesteBilderGraph}
    }
    
    \caption[BilinearBikubisch]{Vergleich Bilinear und Bikubisch}
\end{figure}
Test \ref{DatenbankGraph} wurde von $10000$ bis $10000000$ Bilder durchgeführt. Die Zeit erhöht sich um 1250ns pro Bild. Die Arbeitsspeichernutzung erhöht sich um 87Byte pro Bild. Die Arbeitsspeichernutzung ist dabei auch stark von der Länge des Speicherortes des Bildes abhängig.
\newline
Test \ref{DurchschnittlicheFarbeGraph} wurde von $1000$ bis $25000$ Bilder durchegführt. Die Bilder hatten eine größe von 500x500px. Die Laufzeit für ein Bild liegt bei 1,09ms. Für jedes weitere Bild werden 88kByte verbraucht. Beide Werte verhalten sicht linear zu der Anzahl der Bilder.
\newline

ÜBERPRÜFEN!!!!

Das Skalieren der Bilder in \ref{SkalierenGraph} wurde jeweils mit einem Bilinearen Algorithmus und einem Bikubischen Algorithmus durchgeführt. Dazu wurden 500x500px große Bilder auf 100x70px scaliert. Der Test wurde mit $1000$ Bildern gestartet und mit $12500$ beendet. Es ist zu erkennen, dass nur ein geringer Laufzeitunterschied von $\Delta t = 4,2ms - 3,7ms = 0,5ms$ zwischen den beiden Algorithmen besteht. Dieser geringe unterschied kommt daher, dass beide Algorithmen sich sehr ähnlich verhalten. Der Unterscheid liegt lediglich bei der komplexeren Berechnung in dem Bikubischem Algorithmus, welche nur einen Teil der Laufzeit beansprucht.
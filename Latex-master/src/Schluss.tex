\section{Schluss}
Das Bearbeiten dieser Arbeit hat mich vieles gelehrt. Nicht nur habe ich ein tieferes Verständnis für das Programmieren bekommen, sondern auch das Managen von großen Projekten\footnote{Das Projekt betrug über 5540 Zeilen Code in 49 Klassen.}. Jedoch war der ``Do it yourself'' Weg nicht immer leicht und es war nicht immer einfach etwas richtiges zu finden. Besonders weil das Thema Prozesse nicht das Bekannteste im Internet ist. Meine Anfangs naive Einstellung ``Threads sind nur da um mehr Kerne zu nutzen.'' hat sich als Falsch herausgestellt. Die Enorme Komplexität hinter den wohl wichtigsten Prinzipien für die Moderne Welt ist nicht zu überschätzen. Ich habe mich schließlich im Linux Kernel Code wiedergefunden um zu verstehen, wie ein \textit{Scheduler} funktioniert.

\subsection{Aussicht}
In meinen Zukünftigen Projekten möchte ich meine Nutzung von Threads weiterhin verbessern. Aber auch möchte ich noch mehr Funktionen zu meinem Programm hinzufügen. Ein weiterer Meilenstein, den ich mir erarbeiten möchte ist die Nutzung der Grafikkarte. Die Enormen Rechenkapazitäten, welche in der Grafikkarte verborgen ist, würde mir neue Türen für noch größere Projekte öffnen.
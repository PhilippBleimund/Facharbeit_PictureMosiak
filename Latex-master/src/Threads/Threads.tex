\section{Threads und Prozesse}

\subsection{Einleitung}
Prozesse sind die Ausführung eines Programms auf dem Prozessor. Jedoch kann ein Prozessor maximal ein Prozess gleichzeitig ausführen. Um Verwirrung zu beseitigen möchte ich darauf hinweisen, dass selbst moderne Prozessoren nicht in der Lage sind mehrere Prozesse auszuführen. Diese ``Illusion'' wird erzeugt, da ein Prozessor(Bauteil) mehre Kerne hat. Diese Kerne sind die eigentlichen Prozessoren. In Zukunft werde ich den Begriff Kerne nutzen um die Unterscheidung zu erleichtern. Um trotzdem mehrere Prozesse gleichzeitig zu bearbeiten, werden den einzelnen Kernen die Prozesse für nur wenige Millisekunden zugeordnet. Diese nennt man auch Virtuelle Threads. Jeder Virtuelle Thread kann einem realem Kern zugeordnet werden. Jedoch wird nicht jeder Process gleich lange einem Kern zugeordnet. Die Prozesse konkurrieren um ihre Zeit. Denn schließlich soll mein Computerspiel nicht die gleiche Zeit bekommen, wie meine Stoppuhr app. Diese wechsel zwischen den einzelnen Prozessen nennt man auch Kontextwechsel. Der Kontext des Kerns ändert sich demnach.

\begin{figure}[h]
    \centering
    \begin{tikzpicture}[auto, thick, node distance=2cm, >=triangle 45]
        \draw
        node at(0,0)[name=start1]{}
        node [sum, name=VThread1, right of=start1]{$V_1$}
        node [sum, name=VThread2, right of=VThread1]{$V_2$}
        node [sum, name=VThread3, right of=VThread2]{$V_3$}
        node [sum, name=VThread4, right of=VThread3]{$V_4$}
        node [sum, name=VThread5, right of=VThread4]{$V_5$}
        node [sum, name=VThread6, right of=VThread5]{$V_6$}
        node [sum, name=VThread7, right of=VThread6]{$V_7$}
        node [sum, name=VThread8, right of=VThread7]{$V_8$};
        \draw
        node at(1.7,-3)[name=start2]{}
        node [block, name=Kern1, right of=start2]{$K_1$}
        node [block, name=Kern2, right of=Kern1]{$K_2$}
        node [block, name=Kern3, right of=Kern2]{$K_3$}
        node [block, name=Kern4, right of=Kern3]{$K_4$};
        \draw[->](VThread1) -- node{} (Kern1.north);
        \draw[->](VThread3) -- node{} (Kern2.north);
        \draw[->](VThread4) -- node{} (Kern3.north);
        \draw[->](VThread7) -- node{} (Kern4.north);
    \end{tikzpicture}
    \caption{Aufteilung von Virtuellen Threads auf Kernel}
\end{figure}

\subsection{Aufbau von Prozessen}

Prozesse müssen jedoch noch ein wenig mehr als nur ein Stück code besitzen, um aktiv zu werden. Generell kann man sagen, dass Prozesse aus 7 Elementen bestehen.
Diese nennt man Prozesskontext. Innerhalb des Prozesskontextes gibt es noch den Hardwarekontext.
\begin{itemize}
    \setlength\itemsep{0pt}
    \item Das auszuführende Programm
    \item Die Daten des Programmes. Umfasst etwa die Globalen variablen.
    \item Einen Stack. Ein Stack funktioniert nach dem push und pop verfahren und speichert die lokalen Variablen für einen schnelleren zugriff.
    \item Kernelstack. Umfasst die Systemaufrufe des Prozesses.
          \begin{itemize}
              \item CPU Register. Kann in den meisten Fällen nur ein Befehl speichern (64bit Prozessor = 64bits im Register)
              \item MMU Register, dass den zugriff auf den Arbeitsspeicher verwaltet.
          \end{itemize}
\end{itemize}

Da ein Prozess viele Kontextwechsel durchleben wird, muss das Betriebssystem bestimmte Register speichern. Dazu gehört aus dem Hardwarekontext:

\begin{itemize}
    \setlength\itemsep{0pt}
    \item Instruction Pointer - Die Speicheradresse des nächsten Befehls.
    \item Instruction Register - Der aktuelle Befehl.
    \item Stackpointer - Speichert das ende des Stacks.
    \item Basepointer - Speicheradresse des aktuellen Elementes im Stack.
    \item Akkumulator - Speichert, Ergebnisse der ALU
\end{itemize}

Dies sind die wichtigsten Informationen, um die Rechenoperationen weiterführen zu können. Das Betriebssystem braucht jedoch noch weitere Informationen über einen Prozess. Diese werden auch Systemkontext genannt. Die wichtigsten davon sind:

\begin{itemize}
    \setlength\itemsep{0pt}
    \item Ort in der Prozesstabelle
    \item PID - Prozessnummer
    \item Prozesszustand
    \item Priorität
    \item Eltern- oder Kindprozesse
    \item Zugriffsrechte - Linux: -20 bis 19; Windows: Rechte werden einzeln zugeteilt
    \item Erlaubte Ressourcenmengen - Bsp. Maximaler RAM verbrauch
    \item Verwendete Dateien - Um zu verhindern, dass mehre Prozesse an einer Datei arbeiten
    \item Zugeordnete Geräte - Maus, Tastatur, ...
\end{itemize}

Mithilfe der Prozesstabelle kann das Betriebssystem die einzelnen Prozesse speichern. In dieser werden Prozesskontrollblöcke gespeichert, welche den Hardwarekontext und Systemkontext beinhaltet. Bei einem Kontextwechsel wird der Prozesskontext aus der Prozesstabelle wieder hergestellt.

\newpage

\subsection{Verwalten der Prozesse}

Jedes Betriebssystem muss einen weg haben, um effektiv die Kontextwechsel der Prozesse durchführen zu können. Dazu wird in den meisten Fällen ein \captionref{Warteschlange Prozesse} verwendet. Auch hat ein Prozess deutlich mehr Zustände als nur \textit{untätig} und \textit{rechnend} in einem modernen Betriebssystem. Dazu wird heutzutage meistens das \captionref{Prozessmodell} oder eine modifizierte Variante. Linux als Beispiel verwendet ein \textit{8-Zustands Prozessmodell}, welches das Modell mit einem \textit{kernel rechnend} Zustand erweitert.

\begin{figure}[h]
    \centering
    \begin{tikzpicture}[y=-1cm, scale=0.8]
    \pgfmathsetmacro\start{2}
    \pgfmathsetmacro\endValue{5}
    \draw[very thick, -{Stealth[scale=1.5]}](-2,1.5) -- (\start,1.5);

    %Warteschlange
    \foreach \x in {0,...,\endValue}{
        \filldraw[very thick, draw = black, fill = black!10](\start+\x, 1) rectangle (\start+\x+1, 2);
    }

    \pgfmathsetmacro\middleLine{\start+((\endValue+1)/2)}
    \node[align=center] at (\middleLine, 0.7){\small Warteschlange \footnotesize(bereit)};

    \draw[very thick, -{Stealth[scale=1.5]}](\start+\endValue+1,1.5) -- (\start+\endValue+3,1.5);

    \draw[very thick, draw = black](\start+\endValue+3, 0.7) rectangle (\start+\endValue+3+2.5, 2.3) node[pos=.5] {CPU};

    %Linien
    \pgfmathsetmacro\middleCPU{\start+\endValue+3+1.25}
    \pgfmathsetmacro\endLine{\start+\endValue+1}

    \draw[very thick](\middleCPU, 2.3) -- (\middleCPU, 9);
    \draw[very thick, -{Stealth[scale=1.5]}](-1, 9) -- (-1, 1.5);

    %Entzug der CPU
    \draw[very thick, -{Stealth[scale=1.5]}](\middleCPU, 3) -- (-1, 3) node [midway, above] {\small Entzug \footnotesize (timeout)};

    %Event 1
    \draw[very thick, -{Stealth[scale=1.5]}](\middleCPU, 4.5) -- (\endLine, 4.5) node [midway, above] {\small timeout};
    \foreach \x in {0,...,\endValue}{
        \filldraw[very thick, draw = black, fill = black!10](\start+\x, 4) rectangle (\start+\x+1, 5) node[pos=.5] {\small 1};
    }
    \node[align=center] at (\middleLine, 3.7){\small Warteschlange \footnotesize(blockiert)};
    \draw[very thick](\start, 4.5) -- (-1, 4.5) node [midway, above, align=center] {\small Event 1\\\small eingetreten};

    %Event 2
    \draw[very thick, -{Stealth[scale=1.5]}](\middleCPU, 6.5) -- (\endLine, 6.5) node [midway, above] {\small timeout};
    \foreach \x in {0,...,\endValue}{
        \filldraw[very thick, draw = black, fill = black!10](\start+\x, 6) rectangle (\start+\x+1, 7) node[pos=.5] {\small 2};
    }
    \node[align=center] at (\middleLine, 5.7){\small Warteschlange \footnotesize(blockiert)};
    \draw[very thick](\start, 6.5) -- (-1, 6.5) node [midway, above, align=center] {\small Event 2\\\small eingetreten};

    %Event n
    \draw[very thick, -{Stealth[scale=1.5]}](\middleCPU, 9) -- (\endLine, 9) node [midway, above] {\small timeout};
    \foreach \x in {0,...,\endValue}{
        \filldraw[very thick, draw = black, fill = black!10](\start+\x, 8.5) rectangle (\start+\x+1, 9.5) node[pos=.5] {\small n};
    }
    \node[align=center] at (\middleLine, 8.2){\small Warteschlange \footnotesize(blockiert)};
    \draw[very thick](\start, 9) -- (-1, 9) node [midway, above, align=center] {\small Event n\\\small eingetreten};
\end{tikzpicture}
    \caption{Warteschlangen System}
    \label{Warteschlange Prozesse}
\end{figure}

\begin{figure}[h]
    \centering
    \begin{tikzpicture}
    %neu
    \draw[very thick, -{Stealth[scale=1.5]}](7, 1) -- (7, 2);
    \filldraw[very thick, draw = black, fill = black!10](6, 3) rectangle (8, 2) node[pos=.5] {\small neu};

    %bereit
    \draw[very thick, -{Stealth[scale=1.5]}](7, 3) -- (7, 4);
    \filldraw[very thick, draw = black, fill = black!10](6, 5) rectangle (8, 4) node[pos=.5] {\small bereit};
    \draw[very thick, -{Stealth[scale=1.5]}](8, 4.3) -- (11, 4.3) node [midway, below] {\small timeout};
    \draw[very thick, -{Stealth[scale=1.5]}](6, 4.7) -- (3, 4.7) node [midway, above] {\small suspendieren};

    %blockiert
    \draw[very thick, -{Stealth[scale=1.5]}](7, 6.5) -- (7, 5) node [midway, anchor=west, align=center] {\small Ereignis\\\small eingetreten};
    \filldraw[very thick, draw = black, fill = black!10](6, 7.5) rectangle (8, 6.5) node[pos=.5] {\small blockiert};
    \draw[very thick, -{Stealth[scale=1.5]}](6, 7.2) -- (3, 7.2) node [midway, above] {\small suspendieren};
    
    %beendet
    \filldraw[very thick, draw = black, fill = black!10](11, 3) rectangle (13, 2) node[pos=.5] {\small beendet};
    \draw[very thick, -{Stealth[scale=1.5]}](12, 2) -- (12, 1);

    %rechnend
    \draw[very thick, -{Stealth[scale=1.5]}](12, 4) -- (12, 3);
    \filldraw[very thick, draw = black, fill = black!10](11, 5) rectangle (13, 4) node[pos=.5] {\small rechnend};
    \draw[very thick, -{Stealth[scale=1.5]}](11, 4.7) -- (8, 4.7) node [midway, above] {\small dispatch};
    \draw[very thick](11.5, 5) -- (11.5, 7);
    \draw[very thick, -{Stealth[scale=1.5]}](11.5, 7) -- (8, 7) node [midway, above, align=center] {\small Warten auf \\\small Ereignis};
    \draw[very thick](12.5, 5) -- (12.5, 9);
    \draw[very thick](12.5, 9) -- (0, 9) node [midway, above] {\small suspendieren};
    \draw[very thick](0, 9) -- (0, 4.5);
    \draw[very thick, -{Stealth[scale=1.5]}](0, 4.5) -- (1, 4.5);

    %bereit suspendiert
    \filldraw[very thick, draw = black, fill = black!10](1, 5) rectangle (3, 4) node[pos=.5, align=center] {\small bereit\\\small suspendiert};
    \draw[very thick, -{Stealth[scale=1.5]}](3, 4.3) -- (6, 4.3) node [midway, below] {\small aktivieren};

    %blockiert suspendiert
    \filldraw[very thick, draw = black, fill = black!10](1, 7.5) rectangle (3, 6.5) node[pos=.5, align=center] {\small blockiert\\\small suspendiert};
    \draw[very thick, -{Stealth[scale=1.5]}](2, 6.5) -- (2, 5) node [midway, anchor=east, align=center] {\small Ereignis\\\small eingetreten};
    \draw[very thick, -{Stealth[scale=1.5]}](3, 6.8) -- (6, 6.8) node [midway, below] {\small aktivieren};
\end{tikzpicture}
    \caption{7-Zustands Prozessmodell}
    \label{Prozessmodell}
\end{figure}

\newpage

Wie in der Einleitung schon angesprochen sind die Zustände \textit{bereit} und \textit{rechnend} die wichtigsten Zustände. Mit diesen alleine könnte ein Betriebssystem funktionieren. Es gäbe dazu dann nur eine Warteschlange, in der sich alle Prozesse des Zustandes \textit{bereit} befinden. Idealer Weise implementiert der \textit{Scheduler}\footnote{Programm zum Managen der Warteschlangen.}  einen Algorithmus, welcher die Priorität der Prozesse berücksichtigt. Wie schon erwähnt muss sich der \textit{Dispatcher}\footnote{Programm zum ausführen der Prozesswechsel.} um noch weitere Zustände kümmern. Diese und ihre Beziehungen sind in Grafik \ref{Prozessmodell} zu finden. Zwei davon währen \textit{neu} und \textit{beendet}. Diese sind für eine größere Flexibilität nützlich. Mit dem \textit{beendet} Zustand, können Informationen nachträglich von einem fertigen Prozess aufgerufen werden. Der Zustand \textit{neu} hat die gemeinsame Funktion mit dem \textit{beendet}-Zustand Ressourcen zu sparen.\newline
Ein Entschiedener Fehler ist es anzunehmen, dass alle Prozesse jederzeit Arbeiten wollen. So könnte ein Programm auf eine Tastatur Eingabe oder andere Ereignisse warten. Um diese Funktionalität bereitstellen zu können gibt es den Zustand \textit{blockiert}. In diesen wechselt ein Prozess nach den berechnen und kann aus diesen sich wieder in die Warteschlange der bereiten Prozesse einordnen. In Grafik \ref{Warteschlange Prozesse} werden unterschiedliche Warteschlangen für unterschiedliche Ereignisse erzeugt. Dieses Vorgehen hat den Vorteil gegenüber einer einzelnen ``blockiert-Warteschlange'', dass häufig genutzte Events wie Tastenanschläge nicht von seltenen Events beeinträchtigt werden.\newline
Da es sehr schnell zu vielen Prozessen kommen kann, wird mit den Zuständen \textit{blockiert suspendiert} und \textit{bereit suspendiert} eine Möglichkeit geschaffen, selten genutzte Prozesse aus dem Arbeitsspeicher in den Massenspeicher\footnote{Spezielle Partitionen auf einer Festplatte. Auch \textit{swap} genannt.} zu verschieben. Wie die Namen schon Implizieren Prozesse in den Zuständen \textit{blockiert} und \textit{bereit} jeweils suspendiert und aktiviert werden. Für zusätzliche Geschwindigkeit, können Prozesse selbst im suspendierten Zustand auf Ereignisse reagieren und von \textit{blockiert suspendiert} in \textit{bereit suspendiert} wechseln. Es gibt demnach ein zweites \captionref{Warteschlange Prozesse} für die suspendierten Prozesse. Dieses beinhaltet keinen zugriff auf die CPU, sondern kann die Prozesse maximal aktivieren und in den Arbeitsspeicher verschieben.

\newpage

\subsection{Funktionsweise des Schedulers}
Der \textit{Scheduler} ist ein sehr wichtiges und mächtiges Stück Code. Es managed alle anderen Prozesse eines Betriebssystems. Es kann sich die Frage gestellt werden, wie der \textit{Scheduler} ausgeführt wird. Ist er nur ein weiterer Prozess? Dies würde aber implizieren, dass er sich selber Managen würde. Oder wird er auf einem eigenen CPU Kern ausgeführt? Aber Linux läuft doch auch auf einem einzelnem Kern. Die Antwort liegt in der Natur des Kernels.
\medskip
\newline
Der Kernel ist die niedrigste Instanz mit der höchsten Berechtigung in einem System. Nichts steht neben ihm und der CPU. Jedes Programm muss über den Kernel um etwas machen zu können. Der \textit{Scheduler} ist ein Teil des Kernels. Der Kernel ist jedoch kein einzelner Prozess, welcher immer läuft, sondern eine Art Bibliothek. Ein Programm wendet sich an den Kernel und nicht der Kernel an das Programm. Dementsprechend läuft der \textit{Scheduler} nicht dauerhaft, sondern wird nach einer gewissen Zeit getriggert. Der \textit{Scheduler} wird dabei entweder von einem beendeten Process getriggert oder nach einer Zeitunterbrechung. Die Zeitunterbrechung wird dabei von der CPU durch den \textit{programmable interrupt timer (PIT)} erzeugt und im Kernel durch den \textit{timer interrupt handler} aufgefangen, welcher auch den \textit{Scheduler} startet. Es wird dabei zwischen einem \textit{ticked kernel} und \textit{tickless kernel} unterschieden. Bei dem \textit{ticked kernel} ist der Zeitintervall immer gleich wogegen der des \textit{tickless kernels} dynamisch verändert werden kann.
\bigskip
\newline
Der \textit{Scheduler} verwaltet die einzelnen Warteschlangen und entscheidet, wann ein Prozess auf die CPU zugreifen darf. Dabei ist es wichtig die beste Effizienz beizubehalten und trotzdem eine gute Verteilung der Prozesszeit zu ermöglichen. Denn Kontextwechsel sind aufwändig. Bei vielen kleinen Prozessen wird viel Zeit für das speichern der Register und das wiederherstellen eines Prozesses aus der \textit{Prozesstabelle} verwendet. Je länger ein Prozess arbeiten kann, desto effizienter wird die Zeit genutzt. Daher haben sich zwei grundlegende Konzepte des \textit{schedulings} gebildet. Diese Schedulingverfahren sind:
\begin{itemize}
    \item \textit{Nicht-präemptives Scheduling}. Bei diesem ist ein Prozess bis zu seiner Fertigstellung über volle Kontrolle über die CPU. Der \textit{Scheduler} führt erst den Kontextwechsel nach dessen Vollendung aus. Dabei kann es zu Situationen kommen, bei denen ein Prozess nicht Vollendet werden kann. Beispielsweise durch eine Endlosschleife in der Programmierung oder der Entwickler setzt bei der Programmierung das \textit{Präemtives Scheduling} voraus.
    \item \textit{Präemtives Scheduling}. Dieses wird seit Windows 3.x und Mac OS8/9 verwendet. Dabei wird nicht auf die Vollendung eines Prozesses gewartet, sondern kann , und wird in den meisten Fällen, der Prozess der CPU vor Beendung entzogen. Der Vorteil ist, dass viele weitere Prozesse ``gleichzeitig'' arbeiten können, ohne dass der Nutzer das ``Einfrieren'' anderer Prozesse erfährt. Der Nachteil dabei ist, dass die Kontextwechsel viel Zeit in Anspruch nehmen. Somit haben die Prozesse weniger Arbeitszeit und die gesamte Leistung der CPU sinkt etwas. Da die Vorteile der größeren Freiheit der leicht verringerten Leistung überwiegen wird dieses Verfahren in den meisten modernen Betriebssysteme verwendet.
\end{itemize}

Auf diesen zwei grundlegenden Systemen haben sich weitere \textit{Scheduling} Verfahren entwickelt, welche das Verwalten der Warteschlange implementieren. Um ein Ideales System zu erschaffen, müssen bestimmte Punkte berücksichtigt werden. Diese lassen sich nicht immer miteinander Vereinbaren und es ist dem Entwickler überlassen, welche er bevorzugt. Diese Kriterien währen wie folgt:
\begin{itemize}
    \item Prozessor-Auslastung - Die Prozessor-Auslastung sollte im Idealfall so hoch wie möglich sein. Damit kein Befehlszyklus\footnote{Ein Befehlszyklus ist der kleinste Zeitintervall einer CPU. Befehle können unterschiedlich viele Befehlszyklen brauchen. Je kleiner der Befehlszyklus ist desto höher ist die Herz Anzahl einer CPU} verschwendet wird.
    \item Antwortzeit - Die Zeit, die vergeht, bis die erste Antwort eines Prozesses nach Anfrage ankommt.
    \item Durchlaufzeit - Die Zeit, die vergeht, bis ein Prozess nach Einreichung beendet ist.
    \item Durchsatz - Wie viele Prozesse in einem vorgegebenen Intervall, beendet werden. Der Intervall kann je nach Anwendungsfall variieren.
    \item Wartezeit - Die Zeit, die ein Prozess in der bereit-Warteschlange verbringt, bis er zugriff auf die CPU bekommt.
    \item Fairness - Die Fairness eines Verfahren bestimmt, wie gut kleine und weniger priorisierte Prozesse eine Chance haben Prozesszeit zu erhalten.
\end{itemize}

Im Folgenden werde ich zwei \textit{Scheduling} Verfahren vorstellen und diese auf die genannten Kriterien bewerten.
\newpage

\subsubsection{Completetely Fair Scheduling}
\textit{Completetely Fair Scheduling}\footnote{\textit{Completetely Fair Scheduling (CFS)} wurde 2007 das erste mal von Ingo Molnar im Linux Kernel eingearbeitet.} ist eine, wie der Name schon impliziert, eine Form des \textit{Ideal Fair Scheduling}. Dieses besagt, dass versucht wird die Prozesse gleich lange Arbeiten zu lassen. In den Tabellen \ref{Fair Scheduling} sind vier Prozesse dargestellt. Alle haben die selbe Priorität. In dem Szenario hat jeder Zeit-Quant($Q_1$, $Q_2$, \dots) eine Zeitspanne von 4ms. In den ersten vier Quanten bekommt jeder Prozess 1ms. Ab $Q_4$ sind Prozess B und D abgeschlossen und A und C erhalten somit in $Q_5$ jeweils 2ms. Durch das Prinzip werden alle Prozesse gleich behandelt.
\begin{figure}[h]
    \centering
    \begin{tabular}{| l | l |}
        \hline
        Prozess & Ausführungszeit\\
        \hline
        A & 8ms\\
        \hline
        B & 4ms\\
        \hline
        C & 16ms\\
        \hline
        D & 4ms\\
        \hline
    \end{tabular}
    \begin{tabular}{| x{1em} | x{1em} | x{1em} | x{1em} | x{1em} | x{1em} | x{1em} | x{1em} | x{1em} |}
        \hline
        \rowcolor{black!10}
        & $Q_1$ & $Q_2$ & $Q_3$ & $Q_4$ & $Q_5$ & $Q_6$ & $Q_7$ & $Q_8$\\
        \hline
        \rowcolor{black!20}
        A & 1 & 2 & 3 & 4 & 6 & 8 & &\\
        \hline
        \rowcolor{black!10}
        B & 1 & 2 & 3 & 4 & & & &\\
        \hline
        \rowcolor{black!20}
        C & 1 & 2 & 3 & 4 & 6 & 8 & 12 & 16\\
        \hline
        \rowcolor{black!10}
        D & 1 & 2 & 3 & 4 & & & &\\
        \hline
    \end{tabular}
    \captionof{table}{\textit{Fair Scheduling (FS)}}
    \label{Fair Scheduling}
\end{figure}

Im CFS wird das Prinzip des FS nicht hauptsächlich mit Zeitscheiben gelöst. Es wird das Konzept der \textit{vruntime} eingebracht. \textit{Vruntime} bedeutet dabei, wie lange ein Prozessor bereits ausgeführt wurde. CFS berechnet weiterhin unterschiedliche Zeitscheiben für die PIT, jedoch werden diese zur Maximierung der Effizienz genutzt. Prozesse mit komplexen Rechnungen sind effizienter, wenn sie mehr Zeit ohne Kontextwechsel haben. Andere Prozesse wie Tracker schadet der Kontextwechsel nicht allzu sehr. Der \textit{Scheduler} addiert auf die alte \textit{vruntime} die Zeit($\Delta t$) die der Processor lief. Sollte der Prozess noch nicht abgeschlossen sein, wird die neue \textit{vruntime} in einen Rot-Schwarz Baum eingefügt. Ein Rot-Schwarz Baum ist ein Binärbaum, welcher eine selbst ausgleichende Natur besitzt. Somit beträgt die einfüge Laufzeit $\mathcal{O}(\log n)$. CFS wählt als nächsten Prozess, den mit der geringsten \textit{vruntime} aus. Dabei wird immer ein Pointer auf das kleinste Blatt gehalten, um eine Laufzeit von $\mathcal{O}(1)$ zu erhalten, um den Process mit der kleinsten \textit{vruntime} zu finden. Durch das wählen der kleinsten \textit{vruntime} wird das Prinzip des \textit{Fair Schedulings} eingehalten.
\medskip
\newpage
Das Nutzen der \textit{vruntime} als zentralen Wert hat weitere Vorteile. So können die Prioritäten ohne Umwege eingearbeitet werden. Der CFS verwaltet in der \textit{vruntime} drei unterschiedliche \textit{Scheduling Policies}.
\begin{itemize}
    \item \textbf{SCHED\_NORMAL/SCHED\_OTHER}. Die normale Regel für Prozesse.
    \item \textbf{SCHED\_BATCH}. Für Prozesse, die nicht interaktiv sind und den Arbeitsbereich des Nutzers nicht stören wollen.
    \item \textbf{SCHED\_IDLE}. Die niedrigste Regel für Prozesse und sie werden an wenigsten bevorzugt.
\end{itemize}
SCHED\_NORMAL und SCHED\_BATCH sind zusätzlich von den \textit{nice} Werten abhängig. Die \textit{nice} Werte reichen von -20 bis 19 und repräsentieren die Priorität eines Prozesses. Je kleiner der \textit{nice} Wert, desto höher ist die Priorität. Dabei ist der unterschied Linear. Die Verteilung des Prozessors auf zwei Prozesse mit jeweils 11 und 12, währe 55\% und 45\%. Bei Prozessen mit -4 und -5 ist auch 55\% und 45\%. Die neue \textit{vruntime} wird mit Formel \ref{calc_vruntime}\cite{Linux:Torvalds} berechnet.
\begin{align}
    \text{vruntime} \mathrel{+}= \text{delta\_exec} \cdot \frac{\text{weigth}}{\text{lw.weight}}
    \label{calc_vruntime}
\end{align}
Dabei ist \textit{weigth} mit 1024 oder \textit{nice\_0} definiert. \textit{lw.weigth} ist die Priorität als \textit{weigth}. Dieses kann aus dem Array in dem Listing \ref{core.c}\cite{Linux:Torvalds} entnommen werden.
\begin{figure}[h]
    \centering      
    \lstset{
    language=C,
    keywordstyle=\color{blue},
    commentstyle=\color{green},
    }
    \begin{tabular}{c}
        \begin{lstlisting}{Latexkernel core.c}
10886 const int sched_prio_to_weight[40] = {
10887 /* -20 */ 88761, 71755, 56483, 46273, 36291,
10888 /* -15 */ 29154, 23254, 18705, 14949, 11916,
10889 /* -10 */ 9548,  7620,  6100,  4904,  3906,
10890 /*  -5 */ 3121,  2501,  1991,  1586,  1277,
10891 /*   0 */ 1024,  820,   655,   526,   423,
10892 /*   5 */ 335,   272,   215,   172,   137,
10893 /*  10 */ 110,   87,    70,    56,    45,
10894 /*  15 */ 36,    29,    23,    18,    15,
10895 };
        \end{lstlisting}
    \end{tabular}
    \captionof{lstlisting}{Latexkernel core.c{\footnotesize (v5.17-rc3)} \textit{nice} Werte als \textit{weigth}}
    \label{core.c}
\end{figure}
\newpage
\subsection{Threads}

\subsection{Implementation in java}

\subsubsection{ThreadPool}

\subsubsection{Thread sicherheit}

\section{Quellen}
MapBild :
SynchronousJFXFileChooser : https://stackoverflow.com/questions/28920758/javafx-filechooser-in-swing
Answer from Sergei Tachenov
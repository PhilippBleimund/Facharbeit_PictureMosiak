\section{Threads und Prozesse}

\subsection{Einleitung}
Prozesse sind die Ausführung eines Programms auf dem Prozessor. Jedoch kann ein Prozessor maximal ein Prozess gleichzeitig ausführen. Um Verwirrung zu beseitigen möchte ich darauf hinweisen, dass selbst moderne Prozessoren nicht in der Lage sind mehrere Prozesse auszuführen. Diese ``Illusion'' wird erzeugt, da ein Prozessor(Bauteil) mehre Kerne hat. Diese Kerne sind die eigentlichen Prozessoren. In Zukunft werde ich den Begriff Kerne nutzen um die Unterscheidung zu erleichtern. Um trotzdem mehrere Prozesse gleichzeitig zu bearbeiten, werden den einzelnen Kernen die Prozesse für nur wenige Millisekunden zugeordnet. Diese nennt man auch Virtuelle Threads. Jeder Virtuelle Thread kann einem realem Kern zugeordnet werden. Jedoch wird nicht jeder Process gleich lange einem Kern zugeordnet. Die Prozesse konkurrieren um ihre Zeit. Denn schließlich soll mein Computerspiel nicht die gleiche Zeit bekommen, wie meine Stoppuhr app. Diese wechsel zwischen den einzelnen Prozessen nennt man auch Kontextwechsel. Der Kontext des Kerns ändert sich demnach.

\begin{figure}[h]
    \centering
    \begin{tikzpicture}[auto, thick, node distance=2cm, >=triangle 45]
        \draw
        node at(0,0)[name=start1]{}
        node [sum, name=VThread1, right of=start1]{$V_1$}
        node [sum, name=VThread2, right of=VThread1]{$V_2$}
        node [sum, name=VThread3, right of=VThread2]{$V_3$}
        node [sum, name=VThread4, right of=VThread3]{$V_4$}
        node [sum, name=VThread5, right of=VThread4]{$V_5$}
        node [sum, name=VThread6, right of=VThread5]{$V_6$}
        node [sum, name=VThread7, right of=VThread6]{$V_7$}
        node [sum, name=VThread8, right of=VThread7]{$V_8$};
        \draw
        node at(1.7,-3)[name=start2]{}
        node [block, name=Kern1, right of=start2]{$K_1$}
        node [block, name=Kern2, right of=Kern1]{$K_2$}
        node [block, name=Kern3, right of=Kern2]{$K_3$}
        node [block, name=Kern4, right of=Kern3]{$K_4$};
        \draw[->](VThread1) -- node{} (Kern1.north);
        \draw[->](VThread3) -- node{} (Kern2.north);
        \draw[->](VThread4) -- node{} (Kern3.north);
        \draw[->](VThread7) -- node{} (Kern4.north);
    \end{tikzpicture}
    \caption{Aufteilung von Virtuellen Threads auf Kernel}
\end{figure}

\subsection{Aufbau von Prozessen}

Prozesse müssen jedoch noch ein wenig mehr als nur ein Stück code besitzen, um aktiv zu werden. Generell kann man sagen, dass Prozesse aus 7 Elementen bestehen.
Diese nennt man Prozesskontext. Innerhalb des Prozesskontextes gibt es noch den Hardwarekontext.
\begin{itemize}
    \setlength\itemsep{0pt}
    \item Das auszuführende Programm
    \item Die Daten des Programmes. Umfasst etwa die Globalen variablen.
    \item Einen Stack. Ein Stack funktioniert nach dem push und pop verfahren und speichert die lokalen Variablen für einen schnelleren zugriff.
    \item Kernelstack. Umfasst die Systemaufrufe des Prozesses.
          \begin{itemize}
              \item CPU Register. Kann in den meisten Fällen nur ein Befehl speichern (64bit Prozessor = 64bits im Register)
              \item MMU Register, dass den zugriff auf den Arbeitsspeicher verwaltet.
          \end{itemize}
\end{itemize}

Da ein Prozess viele Kontextwechsel durchleben wird, muss das Betriebssystem bestimmte Register speichern. Dazu gehört aus dem Hardwarekontext:

\begin{itemize}
    \setlength\itemsep{0pt}
    \item Instruction Pointer - Die Speicheradresse des nächsten Befehls.
    \item Instruction Register - Der aktuelle Befehl.
    \item Stackpointer - Speichert das ende des Stacks.
    \item Basepointer - Speicheradresse des aktuellen Elementes im Stack.
    \item Akkumulator - Speichert, Ergebnisse der ALU
\end{itemize}

Dies sind die wichtigsten Informationen, um die Rechenoperationen weiterführen zu können. Das Betriebssystem braucht jedoch noch weitere Informationen über einen Prozess. Diese werden auch Systemkontext genannt. Die wichtigsten davon sind:

\begin{itemize}
    \setlength\itemsep{0pt}
    \item Ort in der Prozesstabelle
    \item PID - Prozessnummer
    \item Prozesszustand
    \item Priorität
    \item Eltern- oder Kindprozesse
    \item Zugriffsrechte - Linux: -20 bis 19; Windows: Rechte werden einzeln zugeteilt
    \item Erlaubte Ressourcenmengen - Bsp. Maximaler RAM verbrauch
    \item Verwendete Dateien - Um zu verhindern, dass mehre Prozesse an einer Datei arbeiten
    \item Zugeordnete Geräte - Maus, Tastatur, ...
\end{itemize}

Mithilfe der Prozesstabelle kann das Betriebssystem die einzelnen Prozesse speichern. In dieser werden Prozesskontrollblöcke gespeichert, welche den Hardwarekontext und Systemkontext beinhaltet. Bei einem Kontextwechsel wird der Prozesskontext aus der Prozesstabelle wieder hergestellt.

\newpage

\subsection{Verwalten der Prozesse}

Jedes Betriebssystem muss einen weg haben, um effektiv die Kontextwechsel der Prozesse durchführen zu können. Dazu wird in den meisten Fällen ein \captionref{Warteschlange Prozesse} verwendet. Auch hat ein Prozess deutlich mehr Zustände als nur \textit{untätig} und \textit{rechnend} in einem modernen Betriebssystem. Dazu wird heutzutage meistens das \captionref{Prozessmodell} oder eine modifizierte Variante. Linux als Beispiel verwendet ein \textit{8-Zustands Prozessmodell}, welches das Modell mit einem \textit{kernel rechnend} Zustand erweitert.

\begin{figure}[h]
    \centering
    \begin{tikzpicture}[y=-1cm, scale=0.8]
    \pgfmathsetmacro\start{2}
    \pgfmathsetmacro\endValue{5}
    \draw[very thick, -{Stealth[scale=1.5]}](-2,1.5) -- (\start,1.5);

    %Warteschlange
    \foreach \x in {0,...,\endValue}{
        \filldraw[very thick, draw = black, fill = black!10](\start+\x, 1) rectangle (\start+\x+1, 2);
    }

    \pgfmathsetmacro\middleLine{\start+((\endValue+1)/2)}
    \node[align=center] at (\middleLine, 0.7){\small Warteschlange \footnotesize(bereit)};

    \draw[very thick, -{Stealth[scale=1.5]}](\start+\endValue+1,1.5) -- (\start+\endValue+3,1.5);

    \draw[very thick, draw = black](\start+\endValue+3, 0.7) rectangle (\start+\endValue+3+2.5, 2.3) node[pos=.5] {CPU};

    %Linien
    \pgfmathsetmacro\middleCPU{\start+\endValue+3+1.25}
    \pgfmathsetmacro\endLine{\start+\endValue+1}

    \draw[very thick](\middleCPU, 2.3) -- (\middleCPU, 9);
    \draw[very thick, -{Stealth[scale=1.5]}](-1, 9) -- (-1, 1.5);

    %Entzug der CPU
    \draw[very thick, -{Stealth[scale=1.5]}](\middleCPU, 3) -- (-1, 3) node [midway, above] {\small Entzug \footnotesize (timeout)};

    %Event 1
    \draw[very thick, -{Stealth[scale=1.5]}](\middleCPU, 4.5) -- (\endLine, 4.5) node [midway, above] {\small timeout};
    \foreach \x in {0,...,\endValue}{
        \filldraw[very thick, draw = black, fill = black!10](\start+\x, 4) rectangle (\start+\x+1, 5) node[pos=.5] {\small 1};
    }
    \node[align=center] at (\middleLine, 3.7){\small Warteschlange \footnotesize(blockiert)};
    \draw[very thick](\start, 4.5) -- (-1, 4.5) node [midway, above, align=center] {\small Event 1\\\small eingetreten};

    %Event 2
    \draw[very thick, -{Stealth[scale=1.5]}](\middleCPU, 6.5) -- (\endLine, 6.5) node [midway, above] {\small timeout};
    \foreach \x in {0,...,\endValue}{
        \filldraw[very thick, draw = black, fill = black!10](\start+\x, 6) rectangle (\start+\x+1, 7) node[pos=.5] {\small 2};
    }
    \node[align=center] at (\middleLine, 5.7){\small Warteschlange \footnotesize(blockiert)};
    \draw[very thick](\start, 6.5) -- (-1, 6.5) node [midway, above, align=center] {\small Event 2\\\small eingetreten};

    %Event n
    \draw[very thick, -{Stealth[scale=1.5]}](\middleCPU, 9) -- (\endLine, 9) node [midway, above] {\small timeout};
    \foreach \x in {0,...,\endValue}{
        \filldraw[very thick, draw = black, fill = black!10](\start+\x, 8.5) rectangle (\start+\x+1, 9.5) node[pos=.5] {\small n};
    }
    \node[align=center] at (\middleLine, 8.2){\small Warteschlange \footnotesize(blockiert)};
    \draw[very thick](\start, 9) -- (-1, 9) node [midway, above, align=center] {\small Event n\\\small eingetreten};
\end{tikzpicture}
    \caption{Warteschlangen System}
    \label{Warteschlange Prozesse}
\end{figure}

\begin{figure}[h]
    \centering
    \begin{tikzpicture}
    %neu
    \draw[very thick, -{Stealth[scale=1.5]}](7, 1) -- (7, 2);
    \filldraw[very thick, draw = black, fill = black!10](6, 3) rectangle (8, 2) node[pos=.5] {\small neu};

    %bereit
    \draw[very thick, -{Stealth[scale=1.5]}](7, 3) -- (7, 4);
    \filldraw[very thick, draw = black, fill = black!10](6, 5) rectangle (8, 4) node[pos=.5] {\small bereit};
    \draw[very thick, -{Stealth[scale=1.5]}](8, 4.3) -- (11, 4.3) node [midway, below] {\small timeout};
    \draw[very thick, -{Stealth[scale=1.5]}](6, 4.7) -- (3, 4.7) node [midway, above] {\small suspendieren};

    %blockiert
    \draw[very thick, -{Stealth[scale=1.5]}](7, 6.5) -- (7, 5) node [midway, anchor=west, align=center] {\small Ereignis\\\small eingetreten};
    \filldraw[very thick, draw = black, fill = black!10](6, 7.5) rectangle (8, 6.5) node[pos=.5] {\small blockiert};
    \draw[very thick, -{Stealth[scale=1.5]}](6, 7.2) -- (3, 7.2) node [midway, above] {\small suspendieren};
    
    %beendet
    \filldraw[very thick, draw = black, fill = black!10](11, 3) rectangle (13, 2) node[pos=.5] {\small beendet};
    \draw[very thick, -{Stealth[scale=1.5]}](12, 2) -- (12, 1);

    %rechnend
    \draw[very thick, -{Stealth[scale=1.5]}](12, 4) -- (12, 3);
    \filldraw[very thick, draw = black, fill = black!10](11, 5) rectangle (13, 4) node[pos=.5] {\small rechnend};
    \draw[very thick, -{Stealth[scale=1.5]}](11, 4.7) -- (8, 4.7) node [midway, above] {\small dispatch};
    \draw[very thick](11.5, 5) -- (11.5, 7);
    \draw[very thick, -{Stealth[scale=1.5]}](11.5, 7) -- (8, 7) node [midway, above, align=center] {\small Warten auf \\\small Ereignis};
    \draw[very thick](12.5, 5) -- (12.5, 9);
    \draw[very thick](12.5, 9) -- (0, 9) node [midway, above] {\small suspendieren};
    \draw[very thick](0, 9) -- (0, 4.5);
    \draw[very thick, -{Stealth[scale=1.5]}](0, 4.5) -- (1, 4.5);

    %bereit suspendiert
    \filldraw[very thick, draw = black, fill = black!10](1, 5) rectangle (3, 4) node[pos=.5, align=center] {\small bereit\\\small suspendiert};
    \draw[very thick, -{Stealth[scale=1.5]}](3, 4.3) -- (6, 4.3) node [midway, below] {\small aktivieren};

    %blockiert suspendiert
    \filldraw[very thick, draw = black, fill = black!10](1, 7.5) rectangle (3, 6.5) node[pos=.5, align=center] {\small blockiert\\\small suspendiert};
    \draw[very thick, -{Stealth[scale=1.5]}](2, 6.5) -- (2, 5) node [midway, anchor=east, align=center] {\small Ereignis\\\small eingetreten};
    \draw[very thick, -{Stealth[scale=1.5]}](3, 6.8) -- (6, 6.8) node [midway, below] {\small aktivieren};
\end{tikzpicture}
    \caption{7-Zustands Prozessmodell}
    \label{Prozessmodell}
\end{figure}

\newpage

Wie in der Einleitung schon angesprochen sind die Zustände \textit{bereit} und \textit{rechnend} die wichtigsten Zustände. Mit diesen alleine könnte ein Betriebssystem funktionieren. Es gäbe dazu dann nur eine Warteschlange, in der sich alle Prozesse des Zustandes \textit{bereit} befinden. Idealer Weise implementiert der \textit{Scheduler}\footnote{Programm zum Managen der Warteschlangen.}  einen Algorithmus, welcher die Priorität der Prozesse berücksichtigt. Wie schon erwähnt muss sich der \textit{Dispatcher}\footnote{Programm zum ausführen der Prozesswechsel.} um noch weitere Zustände kümmern. Diese und ihre Beziehungen sind in Grafik \ref{Prozessmodell} zu finden. Zwei davon währen \textit{neu} und \textit{beendet}. Diese sind für eine größere Flexibilität nützlich. Mit dem \textit{beendet} Zustand, können Informationen nachträglich von einem fertigen Prozess aufgerufen werden. Der Zustand \textit{neu} hat die gemeinsame Funktion mit dem \textit{beendet}-Zustand Ressourcen zu sparen.\newline
Ein Entschiedener Fehler ist es anzunehmen, dass alle Prozesse jederzeit Arbeiten wollen. So könnte ein Programm auf eine Tastatur Eingabe oder andere Ereignisse warten. Um diese Funktionalität bereitstellen zu können gibt es den Zustand \textit{blockiert}. In diesen wechselt ein Prozess nach den berechnen und kann aus diesen sich wieder in die Warteschlange der bereiten Prozesse einordnen. In Grafik \ref{Warteschlange Prozesse} werden unterschiedliche Warteschlangen für unterschiedliche Ereignisse erzeugt. Dieses Vorgehen hat den Vorteil gegenüber einer einzelnen ``blockiert-Warteschlange'', dass häufig genutzte Events wie Tastenanschläge nicht von seltenen Events beeinträchtigt werden.\newline
Da es sehr schnell zu vielen Prozessen kommen kann, wird mit den Zuständen \textit{blockiert suspendiert} und \textit{bereit suspendiert} eine Möglichkeit geschaffen, selten genutzte Prozesse aus dem Arbeitsspeicher in den Massenspeicher\footnote{Spezielle Partitionen auf einer Festplatte. Auch \textit{swap} genannt.} zu verschieben. Wie die Namen schon Implizieren Prozesse in den Zuständen \textit{blockiert} und \textit{bereit} jeweils suspendiert und aktiviert werden. Für zusätzliche Geschwindigkeit, können Prozesse selbst im suspendierten Zustand auf Ereignisse reagieren und von \textit{blockiert suspendiert} in \textit{bereit suspendiert} wechseln. Es gibt demnach ein zweites \captionref{Warteschlange Prozesse} für die suspendierten Prozesse. Dieses beinhaltet keinen zugriff auf die CPU, sondern kann die Prozesse maximal aktivieren und in den Arbeitsspeicher verschieben.

\newpage

\subsubsection{Funktionsweise des Schedulers}
\lipsum

\subsection{Threads}
\lipsum

\subsection{Implementation in java}
\lipsum

\subsubsection{ThreadPool}
\lipsum[3-5]

\subsubsection{Thread sicherheit}
\lipsum[3-5]

\section{Quellen}
MapBild :
SynchronousJFXFileChooser : https://stackoverflow.com/questions/28920758/javafx-filechooser-in-swing
Answer from Sergei Tachenov
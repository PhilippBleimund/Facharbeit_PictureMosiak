\section{Threads und Prozesse}

\subsection{Einleitung}
Prozesse sind die Ausführung eines Programms auf dem Prozessor. Jedoch kann ein Prozessor maximal ein Prozess gleichzeitig ausführen. Um Verwirrung zu beseitigen möchte ich darauf hinweisen, dass selbst moderne Prozessoren nicht in der Lage sind mehrere Prozesse auszuführen. Diese ``Illusion'' wird erzeugt, da ein Prozessor(Bauteil) mehre Kerne hat. Diese Kerne sind die eigentlichen Prozessoren. In Zukunft werde ich den Begriff Kerne nutzen um die Unterscheidung zu erleichtern. Um trotzdem mehrere Prozesse gleichzeitig zu bearbeiten, werden den einzelnen Kernen die Prozesse für nur wenige Millisekunden zugeordnet. Diese nennt man auch Virtuelle Threads. Jeder Virtuelle Thread kann einem realem Kern zugeordnet werden. Jedoch wird nicht jeder Process gleich lange einem Kern zugeordnet. Die Prozesse konkurrieren um ihre Zeit. Denn schließlich soll mein Computerspiel nicht die gleiche Zeit bekommen, wie meine Stoppuhr app. Diese wechsel zwischen den einzelnen Prozessen nennt man auch Kontextwechsel. Der Kontext des Kerns ändert sich demnach.

\begin{figure}[h]
    \centering
    \begin{tikzpicture}[auto, thick, node distance=2cm, >=triangle 45]
        \draw
        node at(0,0)[name=start1]{}
        node [sum, name=VThread1, right of=start1]{$V_1$}
        node [sum, name=VThread2, right of=VThread1]{$V_2$}
        node [sum, name=VThread3, right of=VThread2]{$V_3$}
        node [sum, name=VThread4, right of=VThread3]{$V_4$}
        node [sum, name=VThread5, right of=VThread4]{$V_5$}
        node [sum, name=VThread6, right of=VThread5]{$V_6$}
        node [sum, name=VThread7, right of=VThread6]{$V_7$}
        node [sum, name=VThread8, right of=VThread7]{$V_8$};
        \draw
        node at(1.7,-3)[name=start2]{}
        node [block, name=Kern1, right of=start2]{$K_1$}
        node [block, name=Kern2, right of=Kern1]{$K_2$}
        node [block, name=Kern3, right of=Kern2]{$K_3$}
        node [block, name=Kern4, right of=Kern3]{$K_4$};
        \draw[->](VThread1) -- node{} (Kern1.north);
        \draw[->](VThread3) -- node{} (Kern2.north);
        \draw[->](VThread4) -- node{} (Kern3.north);
        \draw[->](VThread7) -- node{} (Kern4.north);
    \end{tikzpicture}
    \caption{Aufteilung von Virtuellen Threads auf Kernel}
\end{figure}

\subsection{Aufbau von Prozessen}

Prozesse müssen jedoch noch ein wenig mehr als nur ein Stück code besitzen, um aktiv zu werden. Generell kann man sagen, dass Prozesse aus 7 Elementen bestehen.
Diese nennt man Prozesskontext. Innerhalb des Prozesskontextes gibt es noch den Hardwarekontext.
\begin{itemize}
    \setlength\itemsep{0pt}
    \item Das auszuführende Programm
    \item Die Daten des Programmes. Umfasst etwa die Globalen variablen.
    \item Einen Stack. Ein Stack funktioniert nach dem push und pop verfahren und speichert die lokalen Variablen für einen schnelleren zugriff.
    \item Kernelstack. Umfasst die Systemaufrufe des Prozesses.
          \begin{itemize}
              \item CPU Register. Kann in den meisten Fällen nur ein Befehl speichern (64bit Prozessor = 64bits im Register)
              \item MMU Register, dass den zugriff auf den Arbeitsspeicher verwaltet.
          \end{itemize}
\end{itemize}

Da ein Prozess viele Kontextwechsel durchleben wird, muss das Betriebssystem bestimmte Register speichern. Dazu gehört aus dem Hardwarekontext:

\begin{itemize}
    \setlength\itemsep{0pt}
    \item Instruction Pointer - Die Speicheradresse des nächsten Befehls.
    \item Instruction Register - Der aktuelle Befehl.
    \item Stackpointer - Speichert das ende des Stacks.
    \item Basepointer - Speicheradresse des aktuellen Elementes im Stack.
    \item Akkumulator - Speichert, Ergebnisse der ALU
\end{itemize}

Dies sind die wichtigsten Informationen, um die Rechenoperationen weiterführen zu können. Das Betriebssystem braucht jedoch noch weitere Informationen über einen Prozess. Diese werden auch Systemkontext genannt. Die wichtigsten davon sind:

\begin{itemize}
    \setlength\itemsep{0pt}
    \item Ort in der Prozesstabelle
    \item PID - Prozessnummer
    \item Prozesszustand
    \item Priorität
    \item Eltern- oder Kindprozesse
    \item Zugriffsrechte - Linux: -19 bis 19; Windows: Rechte werden einzeln zugeteilt
    \item Erlaubte Ressourcenmengen - Bsp. Maximaler RAM verbrauch
    \item Verwendete Dateien - Um zu verhindern, dass mehre Prozesse an einer Datei arbeiten
    \item Zugeordnete Geräte - Maus, Tastatur, ...
\end{itemize}

Mithilfe der Prozesstabelle kann das Betriebssystem die einzelnen Prozesse speichern. In dieser werden Prozesskontrollblöcke gespeichert, welche den Hardwarekontext und Systemkontext beinhaltet. Bei einem Kontextwechsel wird der Prozesskontext aus der Prozesstabelle wieder hergestellt.

\subsection{Verwalten der Prozesse}
Mithilfe der Prozesstabelle kann das Betriebssystem die einzelnen Prozesse speichern. In dieser werden Prozesskontrollblöcke gespeichert, welche den Hardwarekontext und Systemkontext beinhaltet. Bei einem Kontextwechsel wird der Prozesskontext aus der Prozesstabelle wieder hergestellt.\\
\begin{figure}[!h]
    \centering
    \begin{tikzpicture}[y=-1cm, scale=0.8]
    \pgfmathsetmacro\start{2}
    \pgfmathsetmacro\endValue{5}
    \draw[very thick, -{Stealth[scale=1.5]}](-2,1.5) -- (\start,1.5);

    %Warteschlange
    \foreach \x in {0,...,\endValue}{
        \filldraw[very thick, draw = black, fill = black!10](\start+\x, 1) rectangle (\start+\x+1, 2);
    }

    \pgfmathsetmacro\middleLine{\start+((\endValue+1)/2)}
    \node[align=center] at (\middleLine, 0.7){\small Warteschlange \footnotesize(bereit)};

    \draw[very thick, -{Stealth[scale=1.5]}](\start+\endValue+1,1.5) -- (\start+\endValue+3,1.5);

    \draw[very thick, draw = black](\start+\endValue+3, 0.7) rectangle (\start+\endValue+3+2.5, 2.3) node[pos=.5] {CPU};

    %Linien
    \pgfmathsetmacro\middleCPU{\start+\endValue+3+1.25}
    \pgfmathsetmacro\endLine{\start+\endValue+1}

    \draw[very thick](\middleCPU, 2.3) -- (\middleCPU, 9);
    \draw[very thick, -{Stealth[scale=1.5]}](-1, 9) -- (-1, 1.5);

    %Entzug der CPU
    \draw[very thick, -{Stealth[scale=1.5]}](\middleCPU, 3) -- (-1, 3) node [midway, above] {\small Entzug \footnotesize (timeout)};

    %Event 1
    \draw[very thick, -{Stealth[scale=1.5]}](\middleCPU, 4.5) -- (\endLine, 4.5) node [midway, above] {\small timeout};
    \foreach \x in {0,...,\endValue}{
        \filldraw[very thick, draw = black, fill = black!10](\start+\x, 4) rectangle (\start+\x+1, 5) node[pos=.5] {\small 1};
    }
    \node[align=center] at (\middleLine, 3.7){\small Warteschlange \footnotesize(blockiert)};
    \draw[very thick](\start, 4.5) -- (-1, 4.5) node [midway, above, align=center] {\small Event 1\\\small eingetreten};

    %Event 2
    \draw[very thick, -{Stealth[scale=1.5]}](\middleCPU, 6.5) -- (\endLine, 6.5) node [midway, above] {\small timeout};
    \foreach \x in {0,...,\endValue}{
        \filldraw[very thick, draw = black, fill = black!10](\start+\x, 6) rectangle (\start+\x+1, 7) node[pos=.5] {\small 2};
    }
    \node[align=center] at (\middleLine, 5.7){\small Warteschlange \footnotesize(blockiert)};
    \draw[very thick](\start, 6.5) -- (-1, 6.5) node [midway, above, align=center] {\small Event 2\\\small eingetreten};

    %Event n
    \draw[very thick, -{Stealth[scale=1.5]}](\middleCPU, 9) -- (\endLine, 9) node [midway, above] {\small timeout};
    \foreach \x in {0,...,\endValue}{
        \filldraw[very thick, draw = black, fill = black!10](\start+\x, 8.5) rectangle (\start+\x+1, 9.5) node[pos=.5] {\small n};
    }
    \node[align=center] at (\middleLine, 8.2){\small Warteschlange \footnotesize(blockiert)};
    \draw[very thick](\start, 9) -- (-1, 9) node [midway, above, align=center] {\small Event n\\\small eingetreten};
\end{tikzpicture}
    \caption{Warteschlangen System}
\end{figure}

\begin{figure}[!h]
    \centering
    \begin{tikzpicture}
    %neu
    \draw[very thick, -{Stealth[scale=1.5]}](7, 1) -- (7, 2);
    \filldraw[very thick, draw = black, fill = black!10](6, 3) rectangle (8, 2) node[pos=.5] {\small neu};

    %bereit
    \draw[very thick, -{Stealth[scale=1.5]}](7, 3) -- (7, 4);
    \filldraw[very thick, draw = black, fill = black!10](6, 5) rectangle (8, 4) node[pos=.5] {\small bereit};
    \draw[very thick, -{Stealth[scale=1.5]}](8, 4.3) -- (11, 4.3) node [midway, below] {\small timeout};
    \draw[very thick, -{Stealth[scale=1.5]}](6, 4.7) -- (3, 4.7) node [midway, above] {\small suspendieren};

    %blockiert
    \draw[very thick, -{Stealth[scale=1.5]}](7, 6.5) -- (7, 5) node [midway, anchor=west, align=center] {\small Ereignis\\\small eingetreten};
    \filldraw[very thick, draw = black, fill = black!10](6, 7.5) rectangle (8, 6.5) node[pos=.5] {\small blockiert};
    \draw[very thick, -{Stealth[scale=1.5]}](6, 7.2) -- (3, 7.2) node [midway, above] {\small suspendieren};
    
    %beendet
    \filldraw[very thick, draw = black, fill = black!10](11, 3) rectangle (13, 2) node[pos=.5] {\small beendet};
    \draw[very thick, -{Stealth[scale=1.5]}](12, 2) -- (12, 1);

    %rechnend
    \draw[very thick, -{Stealth[scale=1.5]}](12, 4) -- (12, 3);
    \filldraw[very thick, draw = black, fill = black!10](11, 5) rectangle (13, 4) node[pos=.5] {\small rechnend};
    \draw[very thick, -{Stealth[scale=1.5]}](11, 4.7) -- (8, 4.7) node [midway, above] {\small dispatch};
    \draw[very thick](11.5, 5) -- (11.5, 7);
    \draw[very thick, -{Stealth[scale=1.5]}](11.5, 7) -- (8, 7) node [midway, above, align=center] {\small Warten auf \\\small Ereignis};
    \draw[very thick](12.5, 5) -- (12.5, 9);
    \draw[very thick](12.5, 9) -- (0, 9) node [midway, above] {\small suspendieren};
    \draw[very thick](0, 9) -- (0, 4.5);
    \draw[very thick, -{Stealth[scale=1.5]}](0, 4.5) -- (1, 4.5);

    %bereit suspendiert
    \filldraw[very thick, draw = black, fill = black!10](1, 5) rectangle (3, 4) node[pos=.5, align=center] {\small bereit\\\small suspendiert};
    \draw[very thick, -{Stealth[scale=1.5]}](3, 4.3) -- (6, 4.3) node [midway, below] {\small aktivieren};

    %blockiert suspendiert
    \filldraw[very thick, draw = black, fill = black!10](1, 7.5) rectangle (3, 6.5) node[pos=.5, align=center] {\small blockiert\\\small suspendiert};
    \draw[very thick, -{Stealth[scale=1.5]}](2, 6.5) -- (2, 5) node [midway, anchor=east, align=center] {\small Ereignis\\\small eingetreten};
    \draw[very thick, -{Stealth[scale=1.5]}](3, 6.8) -- (6, 6.8) node [midway, below] {\small aktivieren};
\end{tikzpicture}
    \caption{Warteschlangen System}
\end{figure}


\begin{center}
    \begin{tikzpicture}[auto, thick, node distance=2cm, >=triangle 45]
        \draw
        node at(0,0)[sum, name=start]{}
        node [base, name=bereit, right of=start]{bereit}
        node [base, name=aktiv, node distance=3cm, right of=bereit]{aktiv}
        node [base, name=beendet, node distance=3cm, right of=aktiv]{beendet};
        \draw
        node at(3.5,-2)[base, name=blockiert]{blockiert}
        node [sum, name=friedhof, node distance=2cm, below of=beendet]{};
        \draw[->](start) -- node{} (bereit.west);
        \path[->](bereit.east) edge[bend right=50] node{} (aktiv.west);
        \path[->](aktiv.west) edge[bend right=50] node{} (bereit.east);
        \path[->](aktiv.south) edge[bend left=20] node{} (blockiert.north);
        \path[->](blockiert.north) edge[bend left=20] node{} (bereit.south);
        \draw[->](aktiv.east) -- node{} (beendet.west);
        \draw[->](beendet.south) -- node{} (friedhof);
    \end{tikzpicture}
\end{center}

 \newpage

\subsection{In Windows}
\subsection{Auf der CPU}
\subsection{Implementation in java}
\subsubsection{ThreadPool}
\subsubsection{Thread sicherheit}

\section{Quellen}
MapBild :
SynchronousJFXFileChooser : https://stackoverflow.com/questions/28920758/javafx-filechooser-in-swing
Answer from Sergei Tachenov
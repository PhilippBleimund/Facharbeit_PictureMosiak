\subsection{Idee}
In der genaueren Betrachtung, muss das Programm das eingegebene Bild vereinfachen. Dies wird durch eine Unterteilung des Bildes in Sektoren erreicht. Die Sektoren sind auch die zukünftigen Stellen für die ausgewählten Bilder. Es wird die durchschnittliche Farbe der Sektoren berechnet. Das selbe vorgehen wird auch auf die ausgewählten Bilder angewendet. Dabei ist ein Bild ein Sektor. Das Vorgehen erlaubt es auch, Datenbanken (im JSON-Format) zu erstellen, da sich die durchschnittliche Farbe der ausgewählten Bilder nicht ändern wird. Anschließend wird mit einem Algorithmus die passenden Bilder für die einzelnen Sektoren berechnet. Dabei unterliegt der Algorithmus der Beschränkung, dass der Nutzer wählen kann, wie häufig ein Bild vorkommen darf. Nach dem Berechnen der benötigten Bilder müssen diese für die Verwendung angepasst werden. Es soll schließlich nicht ein 10x10px großer Teil aus einem 4000x3000px Bild verwendet werden. Dazu wird das Bild erst skaliert und schließlich zurechtgeschnitten. Dabei liegt der Fokus darauf, immer die Mitte des Bildes zu treffen. Das Programm arbeitet schrittweise und die nachfolgenden Kapitel stellen die Reihenfolge dar.